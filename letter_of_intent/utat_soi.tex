\documentclass{letter}
\usepackage{hyperref}
\usepackage[scale=0.8]{geometry}

\signature{Leo Wu}
\date{\today}

\begin{document}

\begin{letter}{}

  \opening{To the Executives of the University of Toronto Aerospace Team,}

  As a first-year student hoping to major in Mathematics, Computer Science, and Statistics, I am extremely excited to apply my growing theoretical and technical knowledge to meaningful, interdisciplinary research through the researcher position at the University of Toronto Aerospace Team (UTAT). For me, UTAT's IAC 2026 Bidirectional Technology (BiTs) project is an intriguing opportunity both to contribute to research that tackles the challenges in both the aerospace and climate sectors and to expand my scope of knowledge and capabilities.

  My interest in problem-solving began long before high school and university. In grade 7, I was part of the FIRST Lego League robotics team. There was a lot of pressure. None of us had experience in robotics or programming, and yet we had to compete with high school teams in making a robot that had to find its way through a map using sensors and complete various tasks - all in just a few months. Undeterred, I, along with the whole team, put in extra hours. We poured everything we had into the project: showing up almost an hour before the teacher would arrive, waiting at the door in the cold BC winter, researching the robot parts at home, and building the robot after school. We loved what we did - our raw fire of passion kept us going. Six years later, that fire of passion in me has been transferred to ML and RL. At UTAT, I hope to thrive on the immense time pressure and commit extra hours to the project, just like I had during FIRST Lego League.

My current involvement as a Machine Learning Engineer at the University of Toronto Autonomous Scale Racing (UTASR) team strengthened both my technical skills and my ability to work collaboratively within a cohesive team. The team's goal is to design and optimize machine learning driven autonomous race cars that compete with others in racing around a track. My responsibilities include 1) perfecting a PyTorch-based residual neural network that uses Segment Anything Model 2 (sam2) to automatically mask objects in frames captured by the car's onboard camera, and 2) developing a complete ingestion pipeline that transfers camera input to our neural networks. My works in the club have and continue to challenge me in bettering myself both technically and in working with a team, and I find the experience immensely rewarding. At UTAT, I hope to apply my curiosity, technical knowledge, and collaboration skills that I've honed through UTASR in producing meaningful research for the UTAT team.

Having been an International Baccalaureate Diploma Programme (IBDP) student, I harboured rich experiences in independent research. My extended essay, a 4000-word, two-year-long mandatory independent research component of the IBDP, was on psychology with a focus on obedience, namely that shown in the Milgram experiment. This experience provided me with hands-on research experience, where I worked closely with a supervisor, reviewed existing literature, identified gaps in existing research, and evaluated the credibility and applicability of existing results to various scenarios. I have no doubts that the strong research foundation that this essay gave me - including synthesizing research across vast amounts of literature and conducting original evaluation - would greatly help me in meaningfully contributing to the UTAT BiTs research team.

Should I be accepted, I would actively work towards closing any knowledge gaps between me and what the position requires of me. I believe that my strong work ethic, adaptability, and ability to learn quickly would easily prevent any knowledge gaps from being a barrier for me to contribute significantly to UTAT's research.

Thank you very much for considering my application. I would be incredibly honoured to contribute to BiTs' mission of addressing challenges in both the aerospace and climate sectors.

\closing{Sincerely,}

\end{letter}

\end{document}
